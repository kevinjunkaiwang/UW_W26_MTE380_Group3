\documentclass{article}
\usepackage[margin=1in]{geometry}
\usepackage{amsmath}
\usepackage{booktabs}
\usepackage{siunitx}
\begin{document}

\title{Fuzzy Scheduler Workflow (paper\_fuzzy.py)}
\author{}
\date{}
\maketitle

\section{Overview}
The scheduler follows the structure described in the DCOSS 2020 paper ``A Fuzzy Rule-Based Control System for Fast Line-Following Robots'': two inputs ($X_1$ speed, $X_2$ visible line length) feed six fuzzy rules; the aggregated output is defuzzified to a scalar $X^*$ (range 1--100). The winning output label also maps to a PID gain set and speed cap used by the downstream controller. Membership triangles and PID tuples here are heuristic placeholders; the paper does not publish exact values, so tune them on hardware.

\section{Constructing Inputs from Raw Signals (paper formulas)}
From Section IV of the paper, the augmented decision vector $x(k)$ contains controllable motor parameters (e.g., wheel speeds). The two fuzzy inputs are normalized as:
\begin{align}
X_1 &= \frac{\max_j x_{j,k}}{|X|}\times 100, \quad j\in[1,\ldots,N]\\
X_2 &= \frac{L_k}{|L|}\times 100
\end{align}
where:
\begin{itemize}
\item $x_{j,k}$ is the commanded speed of motor $j$ at iteration $k$; $|X|$ is the maximum motor speed (rpm) capability.
\item $L_k$ is the length of the currently detected line segment (from vision); $|L|$ is the maximum detectable line length for the camera.
\end{itemize}
In this repository, the same normalization is applied conceptually, but the numbers can be synthesized (e.g., \texttt{test.py}) or derived from the webcam line detector:
\begin{itemize}
\item $X_1$: take the current forward speed command (normalized 0--1), multiply by 100.
\item $X_2$: take the fraction of ROI rows containing line pixels (``len\_pct'' in \texttt{vision.py}) as a percentage.
\end{itemize}

\section{Membership Functions}
Triangular membership function:
\[
\mu_{\triangle}(x;a,b,c)=
\begin{cases}
0,& x\le a \text{ or } x\ge c\\
\dfrac{x-a}{b-a},& a<x<b\\
\dfrac{c-x}{c-b},& b\le x<c
\end{cases}
\]

\subsection*{Input $X_1$ (motor speed \%)}
\begin{center}
\begin{tabular}{lll}
\toprule
Label & Triangle $(a,b,c)$ & Notes\\
\midrule
Low    & $(0,0,40)$  & rises to 1 at 0, falls to 0 at 40\\
Medium & $(20,50,80)$& centered at 50\\
High   & $(60,100,100)$& starts at 60, flat to 100\\
\bottomrule
\end{tabular}
\end{center}

\subsection*{Input $X_2$ (line length \%)}
\begin{center}
\begin{tabular}{lll}
\toprule
Label & Triangle $(a,b,c)$ & Notes\\
\midrule
Close & $(0,0,50)$\\
Far   & $(30,100,100)$\\
\bottomrule
\end{tabular}
\end{center}

\subsection*{Output sets (speed level)}
Labels: LC, LF, MC, MF, HC, HF with triangles:
\[
\begin{aligned}
\text{LC}&:(0,10,25),\quad
\text{LF}:(15,30,45),\quad
\text{MC}:(35,50,65),\\
\text{MF}&:(55,70,85),\quad
\text{HC}:(70,85,95),\quad
\text{HF}:(85,100,100).
\end{aligned}
\]

\section{Rule Base (Table~I)}
For each input pair $(X_1, X_2)$:
\begin{center}
\begin{tabular}{lll}
\toprule
$X_1$ & $X_2$ & Output label\\
\midrule
Low    & Close & LC\\
Low    & Far   & LF\\
Medium & Close & MC\\
Medium & Far   & MF\\
High   & Close & HC\\
High   & Far   & HF\\
\bottomrule
\end{tabular}
\end{center}

Rule activation uses fuzzy AND as $\min(\mu_{X_1},\mu_{X_2})$. If multiple rules map to the same output label, fuzzy OR is $\max$ of activations (Mamdani max--min).

\section{Aggregation and Defuzzification}
\begin{enumerate}
\item For each sample $x\in[0,100]$ (101 evenly spaced points; increase density if you need finer centroid accuracy), compute the aggregated output membership:
\[
\mu_{\text{agg}}(x)=\max_{L\in \{\text{labels}\}} \min\big(\mu_{\text{rule}}(L),\, \mu_{L}(x)\big).
\]
\item Centroid defuzzification:
\[
X^* = \frac{\sum_x \mu_{\text{agg}}(x)\, x}{\sum_x \mu_{\text{agg}}(x)}.
\]
\item Quantize and clamp: $X^*_{\text{q}}=\min(100,\max(1,\text{round}(X^*,2)))$.
\end{enumerate}

\section{From $X^*$ to Motor Commands}
\begin{enumerate}
\item Choose the winning label (highest activation) and fetch $(v_{\max}, K_p, K_i, K_d)$ from Table~\ref{tab:pidmap}.
\item Set base speed $v$ such that $|v|\le v_{\max}$ (cap depends on label).
\item Measure lateral error $e$ from the IR array (weighted centroid); compute PID correction
\[
u = K_p e + K_i \int e\,dt + K_d \frac{de}{dt}.
\]
\item Mix to wheel commands (normalized):
\[
u_L = \text{clamp}(v - u,\,-1,\,1),\quad
u_R = \text{clamp}(v + u,\,-1,\,1).
\]
\item Map to TB6612FNG: direction pins from sign of $u_L,u_R$, PWM duty $=|u|\times255$.
\end{enumerate}

\begin{table}[h]
\centering
\begin{tabular}{lcccc}
\toprule
Label & $v_{\max}$ & $K_p$ & $K_i$ & $K_d$\\
\midrule
LC & 0.30 & 0.80 & 0.00 & 0.10\\
LF & 0.35 & 0.70 & 0.00 & 0.10\\
MC & 0.45 & 0.65 & 0.00 & 0.12\\
MF & 0.55 & 0.55 & 0.00 & 0.14\\
HC & 0.65 & 0.45 & 0.00 & 0.16\\
HF & 0.75 & 0.40 & 0.00 & 0.18\\
\bottomrule
\end{tabular}
\caption{PID mapping used in this repo (heuristic), aligned with labels from the paper's Table~I.}
\label{tab:pidmap}
\end{table}

\section{Worked Example (X$_1$=20, X$_2$=30)}
\begin{enumerate}
\item $\mu_{\text{Low}}(20)=0.5$, $\mu_{\text{Close}}(30)=0.4$; other memberships are 0.
\item Rule Low$\land$Close $\rightarrow$ LC activates with $\min(0.5,0.4)=0.4$.
\item Only LC contributes; its triangle $(0,10,25)$ is clipped at 0.4, centroid over $x\in[0,25]$ yields $X^*=12.04$.
\item Label LC wins; select PID $(0.30, 0.80, 0.00, 0.10)$.
\end{enumerate}

\end{document}
